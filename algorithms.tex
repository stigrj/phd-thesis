\chapter{Algorithms}
\section{One-electron systems}
\subsection{Power method}
The simplest procedure to solve the fixed-point problem in Eq.~(\ref{eq:int_schrodinger}) 
is perhaps the power method where the operator is applied iteratively
\begin{align}
    \label{eq:power_iter}
    \tilde{\varphi}^{n} &= -2 \hat{H}^n\big[V\varphi^{n}\big]\\
    \varphi^{n+1} &= \frac{\tilde{\varphi}^{n}}{\|\tilde{\varphi}^{n}\|}
\end{align}
The tilde on the new wavefunction denotes that it is no longer normalized, 
as the operator $\hat{H}$ does not conserve the norm when the eigenvalue 
is not exact.\cite{Kalos} The iteration label on the operator reflects the 
fact that the operator depends on the energy through $\mu^n = \sqrt{-2E^n}$ 
which needs to be updated in each iteration.

Such an iteration sequence $\boldsymbol{x}^{n+1} = f(\boldsymbol{x}^n)$
will converge to the lowest eigenfunction of $f$, provided that $f$ defines
a so-called contraction map. Schneider\cite{Schneider} proves linear convergence
of the wave function and quadratic convergence of the energy for a fixed operator
$f$. 

\subsection{Energy calculation}
The energy of the wavefunction $\varphi$ is formally calculated as the 
expectation value
\begin{equation}
    E = \frac{\langle \varphi| T + V | \varphi\rangle}{\langle \varphi|\varphi\rangle} 
\end{equation}
where $T=-\nabla^2/2$ is the kinetic energy operator. As pointed out 
above, it is desirable to avoid the application of the kinetic operator, 
so, following Harrison et al.\cite{Harrison}, we exploit the fact that 
the operator $2\hat{H} = (T - E)^{-1}$ in Eq.~\ref{eq:int_schrodinger} 
is basically the inverse of the kinetic operator, and extract the energy 
through the application of this operator. Given a wavefunction $\varphi^{n}$ 
and energy $E^{n}$ (this does not have to be the exact energy of $\varphi^{n}$, 
but it must be the energy used in the construction of the operator $\hat{H}^{n}$) 
at one iteration, we can calculate the (exact) energy $E^{n+1}$ of the 
wavefunction $\varphi^{n+1}$ at the next iteration as follows
\begin{align}
    E^{n+1}
    &=	\frac{\langle\tilde{\varphi}^{n+1}| T + V | \tilde{\varphi}^{n+1}\rangle}
	{\langle\tilde{\varphi}^{n+1}|\tilde{\varphi}^{n+1}\rangle}\\
    &=	\frac{\langle\tilde{\varphi}^{n+1}| T - E^{n} | \tilde{\varphi}^{n+1}\rangle}
	{\langle\tilde{\varphi}^{n+1}|\tilde{\varphi}^{n+1}\rangle}
    +	\frac{\langle\tilde{\varphi}^{n+1}| E^{n} + V | \tilde{\varphi}^{n+1}\rangle}
	{\langle\tilde{\varphi}^{n+1}|\tilde{\varphi}^{n+1}\rangle}\\
    &=	\frac{\langle\tilde{\varphi}^{n+1}| T - E^{n} | -2\hat{H}^{n}\big[V\varphi^{n}\big]\rangle}
	{\langle\tilde{\varphi}^{n+1}|\tilde{\varphi}^{n+1}\rangle}
    +	\frac{\langle\tilde{\varphi}^{n+1}| E^{n} + V |\tilde{\varphi}^{n+1}\rangle}
	{\langle\tilde{\varphi}^{n+1}|\tilde{\varphi}^{n+1}\rangle}\\
    &= -\frac{\langle\tilde{\varphi}^{n+1}| V |\varphi^{n}\rangle}
	{\langle\tilde{\varphi}^{n+1}|\tilde{\varphi}^{n+1}\rangle}
    +	E^{n}
    +	\frac{\langle\tilde{\varphi}^{n+1}| V |\tilde{\varphi}^{n+1}\rangle}
	{\langle\tilde{\varphi}^{n+1}|\tilde{\varphi}^{n+1}\rangle}\\
    &= E^{n} + 
	\frac{\langle\tilde{\varphi}^{n+1}| V |\Delta\tilde{\varphi}^{n}\rangle}
	{\langle\tilde{\varphi}^{n+1}|\tilde{\varphi}^{n+1}\rangle}\\
    &= E^{n} + \Delta E^{n}
\end{align}
where $\Delta \tilde{\varphi}^{n} \mydef \tilde{\varphi}^{n+1} - \varphi^{n}$. 
This means that we can calculate the energy update $\Delta E^{n}$ directly from 
the wavefunction update without having to apply the kinetic energy operator, 
provided that the update comes directly from the application of the Helmholtz 
operator $\Delta\tilde{\varphi}^n = -2 \hat{H}^n\big[V\varphi^n\big] - \varphi^n$.

\section{Many-electron systems}
\subsection{Orbital orthogonalization}
As pointed out above it is necessary to enforce orthonormality between the 
occupied orbitals in order to arrive at a true Aufbau solution of the Kohn-Sham 
equations. The naive approach would be to perform one iteration for each orbital 
and then do a Gram-Schmidt orthogonalization of the resulting orbitals, in order 
of increasing energy. This however, will lead to very slow convergence, especially 
for the higher energy orbitals, as the convergence of each orbital is bounded by 
the level of convergence of all lower lying orbitals, whose convergence is again 
dependent on the accuracy of \emph{all} orbitals through the non-linearity of the 
problem. Harrison et al. \cite{Harrison} describes how to use deflation to extract 
multiple eigenfunctions to the Kohn-Sham operator, by recasting the equation for 
each orbital as a ground state problem. Our approach, which was also suggested by 
Harrison et al., is to always make sure that the orbitals diagonalizes the Kohn-Sham 
matrix. In this way the off-diagonal elements of the Kohn-Sham matrix is removed not 
by projection, but rather by rotation within the orbital space, which means that no 
off-diagonal information is thrown away in each iteration, leading to much faster 
convergence. Since we are only working with occupied orbitals, the matix 
diagonalization is considerably cheaper than the corresponding diagonalization 
using high-quality Gaussian basis sets, where the number of virtual orbitals can 
be much larger than the number of occupied. 

The orbital orthonormalization and Kohn-Sham matrix diagonalization can be 
collected into a single orbital transformation. We define the overlap matrix
\begin{equation}
    \tilde{S}_{ij} = \langle \tilde{\varphi}_i | \tilde{\varphi}_j \rangle
\end{equation}
that orthonormalizes the orbitals through the transformation
\begin{equation}
    \bar{\varphi}_i = \sum_{j=1}^N \tilde{S}^{-1/2}_{ij}\tilde{\varphi_j}
\end{equation}
the Kohn-Sham matrix $\tilde{F}$ in the non-orthogonal basis
\begin{equation}
    \label{eq:KS-matrix}
    \tilde{F} = \langle\tilde{\varphi}_i | T + V_{eff} | \tilde{\varphi}_j\rangle
\end{equation}
and the matrix $M$ that diagonalizes the Kohn-Sham matrix $\bar{F} = \tilde{S}^{-1/2}\tilde{F}\tilde{S}^{1/2}$
in the orthonormal basis $\bar{\varphi}_i$
\begin{equation}
    F = M^T\bar{F}M
\end{equation}
where $F$ is a diagonal matrix with the orbital energies on the diagonal. We get the 
total transformation matrix $U = M^T\tilde{S}^{-1/2}$ that orthonormalizes the orbitals 
and diagonalizes the Kohn-Sham matrix
\begin{equation}
    \varphi_i = \sum_{j=1}^N U_{ij} \tilde{\varphi}_j
\end{equation}
with corresponding orbital energies $\epsilon_i = F_{ii}$.

\subsection{Kohn-Sham matrix calculation}
The Kohn-Sham matrix of Eq.~(\ref{eq:KS-matrix}) can be calculated without having to
apply the kinetic energy operator, provided that the orbitals $\tilde{\varphi}_i$ comes 
directly from the application of the Helmholtz operator in Eq.~(\ref{eq:KS-iter}).
We assume that the Kohn-Sham matrix is diagonal at iteration $n$, and that the diagonal
elements $\epsilon_i$ are used as energies in the corresponding Helmholtz operators.
\begin{align}
    \tilde{F}_{ij}^{n+1}
    &=	\langle\tilde{\phi}_i^{n+1} | T + V_{eff}^{n+1}              | \tilde{\phi}_j^{n+1}\rangle\\
    &=	\langle\tilde{\phi}_i^{n+1} | T - \epsilon_j^{n}             | \tilde{\phi}_j^{n+1}\rangle
    +	\langle\tilde{\phi}_i^{n+1} | \epsilon_j^{n} + V_{eff}^{n+1} | \tilde{\phi}_j^{n+1}\rangle\\
    &=	\langle\tilde{\phi}_i^{n+1} | T - \epsilon_j^{n}             | -2\hat{H}^{n}\big[V_{eff}^{n} 
	\phi_j^{n}\big]\rangle
    +	\langle\tilde{\phi}_i^{n+1} | \epsilon_j^{n} + V_{eff}^{n+1} | \tilde{\phi}_j^{n+1}\rangle\\
    &= -\langle\tilde{\phi}_i^{n+1} | V_{eff}^{n}                    | \phi_j^{n}\rangle
    +	\epsilon_j^{n}
	\langle\tilde{\phi}_i^{n+1} | \tilde{\phi}_j^{n+1}\rangle
    +	\langle\tilde{\phi}_i^{n+1} | V_{eff}^{n}+\Delta V_{eff}^{n} | \tilde{\phi}_j^{n+1}\rangle\\
    &=	\epsilon_j^{n}
	\langle\tilde{\phi}_i^{n+1} | \tilde{\phi}_j^{n+1}\rangle
    +	\langle\tilde{\phi}_i^{n+1} | \Delta V_{eff}^{n}             | \tilde{\phi}_j^{n+1}\rangle
    +	\langle\tilde{\phi}_i^{n+1} | V_{eff}^{n+1}                  | \Delta \tilde{\phi}_j^{n}\rangle\\
    \tilde{F}^{n+1}
    &=  \tilde{S}^{n+1}F^{n} + \Delta \tilde{F}^{n}
\end{align}
We see that there are two contributions to the update of the Kohn-Sham matrix, 
one from the orbital update and one from the potential update, and it is important
to include both to speed up convergence.

\subsection{Krylov Accelerated Inexact Newton (KAIN)}
The fixed-point problem in Eq.~(\ref{eq:int_schrodinger}) can be viewed as finding 
the roots of the the following residual function
\begin{equation}
    f(\varphi) =  -2 \hat{H}\big[V\varphi\big] - \varphi
\end{equation}
which can be done using Newton's method
\begin{align}
    \varphi^{n+1}	&= \varphi^n - \big[J(\varphi^n)\big]^{-1}f(\varphi^n)\\
    \label{eq:newton}
		&= \varphi^n - \big[J(\varphi^n)\big]^{-1}\big(-2 \hat{H^n}\big[V\varphi^n\big] - \varphi^n\big)
\end{align}
where $J(\varphi^n)$ is the Jacobian. Comparing Eq.~(\ref{eq:newton}) with 
Eq.~(\ref{eq:power_iter}), we can identify the power method as an inexact 
Newton method where the Jacobian is approximated by $J(\varphi) \approx -1$.
This approximation, however suboptimal, will work as long as $f$ is a strictly 
decaying function (justification?), at least in the vicinity of its root. 
Harrison \cite{KAIN} describes how to make use of the information in the
iterative history (Krylov subspace) to improve the approximation of the 
Jacobian in his Krylov Accelerated Inexact Newton (KAIN) method. The method 
is similar to the more commonly used Direct Inversion of Iterative Subspace 
(DIIS) method of Pulay \cite{DIIS}, but while DIIS is looking for the best 
step within the iterative subspace, KAIN is using the same information to 
extrapolate to a step outside the iterative subspace and is thus considered 
superior to DIIS.
\\
\\
In the KAIN method, the new orbital update is calculated in terms of the $m$ 
latest orbitals and orbital updates
\begin{equation}
    \label{eq:KAIN_update}
    \delta \tilde{\varphi}^n = \Delta\varphi^n + \sum_{j=1}^m c_j \big[(\varphi^{j}-\varphi^{n}) + 
    (\Delta\varphi^{j}-\Delta\varphi^{n})\big] 
\end{equation}
where the coefficients $c_j$ are obtained by solving the following linear 
system $Ac=b$ 
\begin{align}
    A_{i,j} &= \langle \varphi^n -\varphi^i | \Delta\varphi^n - \Delta\varphi^j\rangle\\
    b_{i}   &= \langle \varphi^n -\varphi^i | \Delta\varphi^n\rangle
\end{align}
The size $m$ of the Krylov subspace is without constraints. The larger it is, 
the better is the Krylov update, but also the larger is the linear system. In 
general, the Krylov update will not conserve the norm of the orbital, so an 
additional normalization step should be added at this point.

\subsection{Algorithms}
%\begin{algorithm}
\begin{alltt}
01 Given initial orbital \(\varphi\sp{0}\) and energy \(E\sp{0}\)
02 loop1: while \(\varepsilon > \) threshold
02     Construct Helmholtz operator \(\hat{H}\sp{n}\sb{\mu}\) using \(\mu = \sqrt{-2E\sp{n}}\)
03     Multiply orbital \(\varphi\sp{n}\) with potential
04     Apply Helmholtz operator Eq.(\ref{eq:helmholtz})
05     Calculate orbital update \(\Delta\tilde{\varphi}\sp{n} = \tilde{\varphi}\sp{n+1}-\varphi\sp{n}\)
11     Calculate orbtial error \(\varepsilon^2 = \|\Delta\tilde{\varphi}\sp{n}\|\) 
06     Calculate energy update \(\Delta{E}\sp{n}\) from Eq.(\ref{eq:energy_update})
07     Calculate KAIN updates \(\delta\tilde{\varphi}\sp{n}\) and \(\delta{E}\sp{n}\) from Eq.(\ref{KAIN_update}) 
08     Update orbital \(\tilde{\varphi}\sp{n+1} = \varphi\sp{n} + \delta\tilde{\varphi}\sp{n}\)
09     Update energy \(E\sp{n+1} = E\sp{n}+\delta{E}\sp{n}\)
10     Normalize orbital \(\varphi\sp{n+1} = \tilde{\varphi}\sp{n+1}/\|\tilde{\varphi}\sp{n+1}\|\)
11 end loop
\end{alltt}

There are two important things to note with the KAIN solver when there is more 
than one orbital. Firstly, as the orbital energies are updated in each iteration, 
it might occur that orbitals switch place during the optimization process, if 
ordered by increasing energy. This means that one must make sure that each 
orbital is linked to the correct orbital in the iterative history. Secondly, 
in the case of degeneracies, the matrix diagonalization will not result in a 
unique set of orbitals, as any basis for the eigenspace of the degenerate 
eigenfunctions can orbtained. It is thus important to keep a consistent 
definition of all orbitals as the iteration proceeds. Both of these issues 
can be treated simultaneously by always applying the same orbital rotation 
that diagonalized the Kohn-Sham matrix to the entire iterative history.

The KAIN updates do not conserve the orthogonality between the orbitals, 
and a second orthogonalization process can be added at the end of the cycle, 
although this is not strictly necessary to achieve convergence.
 
With these extensions, the optimization algorithm for many-electron systems 
are as follows 
\begin{alltt}
01 Given initial orbitals \(\varphi\sb{i}\sp{0}\) and energies \(\epsilon\sb{i}\sp{0}\) (assuming diagonal KS matrix)
02 loop1: while \(\varepsilon > \) threshold
03     Calculate electron density Eq.(\ref{eq:molDens})
04     Calculate effective potential Eq.(\ref{eq:effPot})
05     loop2: for each orbital \(i = 1,...,N\)
06         Construct Helmholtz operator \(\hat{H}\sp{n}\sb{\mu}\) using \(\mu = \sqrt{-2\epsilon\sb{i}\sp{n}}\)
07         Multiply orbital \(\varphi\sb{i}\sp{n}\) with effective potential
08         Apply Helmholtz operator Eq.(\ref{eq:helmholtz})
09         Calculate orbital update \(\Delta\tilde{\varphi}\sb{i}\sp{n} = \tilde{\varphi}\sb{i}\sp{n+1}-\varphi\sb{i}\sp{n}\)
10     end loop2
11     Calculate orbtial error \(\varepsilon\sp{2} = \sum\sb{i} \|\Delta\tilde{\varphi}\sb{i}\sp{n+1}\|\) 
12     Calculate effective potential update
13     Calculate KS matrix update from Eq.(\ref{eq:KS_update})
14     Diagonalize KS matrix
15     Rotate KAIN history
16     Calculate KAIN updates \(\delta\tilde{\varphi}\sb{i}\sp{n+1}\) and \(\delta{\epsilon}\sb{i}\sp{n+1}\) from Eq.(\ref{KAIN_update}) 
17     Update orbitals \(\tilde{\varphi}\sb{i}\sp{n+1} = \varphi\sb{i}\sp{n} + \delta\tilde{\varphi}\sb{i}\sp{n}\)
18     Update energies \(\epsilon\sb{i}\sp{n+1} = \epsilon\sb{i}\sp{n} + \delta{\epsilon}\sb{i}\sp{n}\)
19     Orthonormalize orbitals (Gram-Schmidt)
20 end loop1
\end{alltt}

\pagebreak

