\chapter{Introduction}
The work of this thesis has been contributing in the development of the 
program package MRChem, which is a code developed at the University of Tromsø 
\cite{Fossgaard} that is aiming at a fully numerical treatment of molecular 
systems, based on 
Density Functional Theory (DFT). There are currently a huge number of these
program packages available, each with more or less distinct features, and
what separates MRChem from all of these is the choice of basis functions.
While traditional computational chemistry programs use Gaussian type basis sets
for their efficient evaluation of two- and four-electron integrals, MRChem is
based on the multiresolution wavelet basis.\\

\noindent
Wavelet theory is a rather young field of mathematics, first appearing in the
late 1980s. The initial application was in signal theory \cite{Strang} but in
the early 90s, wavelet-based methods started to appear for the solution of
PDEs and integral equations \cite{Beylkin90}\cite{Alpert93}, and in recent 
years for application in electronic structure calculations
\cite{Harrison}\cite{Niklasson}\cite{Arias}.

\subsection*{The Kohn-Sham equations}
In the Kohn-Sham \cite{Kohn-Sham} formulation of DFT the eigenvalue equations 
for the electronic structure can be written
\begin{equation}
	\label{eq:Kohn-Sham}
	\lbrack
	-\frac{1}{2}\nabla^2+V_{eff}(\boldsymbol{r})\rbrack\psi_i(\boldsymbol{r}) =
	\epsilon_i\psi_i(\boldsymbol{r})
\end{equation}
where the effective potential is the collection of three terms
\begin{equation}
	\label{eq:Veff}
	V_{eff}(\boldsymbol{r}) =
	V_{ext}(\boldsymbol{r})+V_{coul}(\boldsymbol{r})+V_{xc}
\end{equation}
where the external potential $V_{ext}$ is usually just the electron-nuclear
attraction, the Coulomb potential $V_{coul}$ is the electron-electron
repulsion and $V_{xc}$ is the exchange-correlation potential which (in
principle) includes all non-classical effects. The functional form of $V_{xc}$
is not known.\\

\noindent
The nuclear charge distribution is a collection of point charges, and the 
nuclear potential has the analytical form
\begin{equation}
	V_{nuc}(\boldsymbol{r}) = -\sum_{\alpha=1}^{N_{nuc}}
	\frac{Z_{\alpha}}{|\boldsymbol{r}-\boldsymbol{r}_{\alpha}|}
\end{equation}
The electronic charge distribution is given by the Kohn-Sham orbitals
\begin{equation}
	\rho(\boldsymbol{r}) = 2\sum_{i=1}^{N_e/2} |\psi_i(\boldsymbol{r})|^2
\end{equation}
assuming a closed shell system with double occupancy. The electronic potential 
is now given as the solution of the Poisson equation
\begin{equation}
	\nabla^2 V_{coul}(\boldsymbol{r}) = 4\pi\rho(\boldsymbol{r})
\end{equation}
where the orbital-dependence of the potential makes eq.(\ref{eq:Kohn-Sham}) a set
of non-linear equations that is usually solved self-consistently. 
The current work will not be concerned with the solution of the Kohn-Sham
equations, but is rather a precursor to this where some building blocks
required for the DFT calculations are prepared, in particular the solution of 
the Poisson equation.

\subsection*{The Poisson equation}
Solving the Poisson equation for an arbitrary charge distribution is a 
non-trivial task, and is of major importance in many fields of science, 
especially in the field of computational chemistry. A huge effort has been put 
into making efficient Poisson solvers, and usual real-space approaches includes
finite difference (FD) and finite element (FE) methods. FD is a a grid-based
method, which is solving the equations iteratively on a discrete grid of 
pointvalues, while FE is expanding the solution in a basis set, usually by
dividing space into cubic cells and allocate a polynomial basis to each cell.\\

\noindent
It is a well-known fact that the electronic density in molecular systems
is rapidly varying in the vicinity of the atomic nuclei, and a usual problem
with real-space methods is that an accurate treatment of the system requires
high resolution of gridpoints (FD) or cells (FE) in the nuclear regions.
Keeping this high resolution uniformly througout space would yield unnecessary
high accuracy in the interatomic regions, and the solution of the Poisson
equation for molecular systems is demanding a \emph{multiresolution} framework 
in order to achieve numerical efficiency.\\

\noindent
There are ways of resolving these issues using multigrid techniques, and a nice
overview of these methods is given by Beck \cite{Beck}, but this thesis is 
concerned with a third way of doing real-space calculations, one where the
multiresolution character is inherent in the theory, namely using wavelet
bases.\\

\noindent
At this point MRChem is basically a Poisson solver. It has the capabilities of 
representing arbitrary functions in the multiwavelet basis, and calculate the 
potential originating from these functions. This is the result of the work by
\cite{Fossgaard}. The current work includes the
implementation of some basic arithmetic operations involving these function
representations, where the space adaptivity of the grid and strict error
control will be important topics. We will also look at some possible 
optimizations of the existing code, where computational efficiency, memory
requirements and linear scaling with respect to system size will be important.
\\

\noindent
The thesis is split in two parts; theory and implementation. The theory part
gives a brief overview of the mathematical theory of multiwavelets, from the
basic concept of multiresolution analysis, to the representation of functions 
and operators in the multiwavelet basis, and ultimately to the solution of the
Poisson equation. The implementation part gives a short description of the
data structures and algorithms used in the MRChem program, and some details of
how the mathematical theory is implemented in practice. Some numerical results
are given to show the capabilities and performances of the code.


\subsection{Multiwavelets}
In the construction of the multiwavelet basis we start with a simple set of 
orthogonal polynomials $\lbrace\phi_j\rbrace_{j=0}^k$ of order $k$ on the unit 
interval. This can be e.g. the Legendre polynomials shifted to the unit interval. 
If this basis turns out to be too crude to accurately describe the function, we 
split the interval and double the number of basis functions by dilating and 
translating the original basis to both subintervals
\begin{equation}
    \phi_{j,l}^1(x) = 2^{1/2}\phi_j(2x - l), \qquad l = 0,1
\end{equation}
The polynomials are defined only on their respective intervals, so basis functions 
at different translations are by construction orthogonal. The splitting procedure 
can be continued until we have reached a scale $n$ where we are satisfied with the 
accuracy of the representation. At this level of refinement the unit interval has 
been split in $2^n$ intervals, each of size $2^{-n}$ containing a dilated and 
translated version of the original $k$-order basis
\begin{equation}
    \label{eq:two_scale}
    \phi_{j,l}^n(x) = 2^{n/2}\phi_j(2^nx - l), \qquad l = 0,\dots,2^n-1
\end{equation}
These are the so-called scaling functions, and they span the scaling space 
$\scalingspace{n}$. This basis can be used to represent any square integrable 
function to any finite accuracy by adjusting the polynomial order $k$ and/or the 
level of refinement $n$. With this construction, it is easy to verify that each 
scaling space is fully contained in all scaling spaces of higher resolution, and 
we have the following relation
\begin{equation}
    \label{eq:MRA}
    \scalingspace{0} \subset \scalingspace{1} \subset \ldots
    \subset \scalingspace{n} \subset \ldots
\end{equation}
In addition to scaling functions, the multiwavelet basis consists of another set 
of piecewise polynomials, the wavelet functions $\psi_{j,l}^n$, that span the 
wavelet space $\waveletspace{n}$. This space is defined as the orthogonal 
complement of $\scalingspace{n}$ in $\scalingspace{n+1}$
\begin{equation}
    \label{eq:wavelet-def}
    \waveletspace{n} \oplus \scalingspace{n} = \scalingspace{n+1}
\end{equation}
The wavelet functions satisfy the same two-scale relation as the scaling functions 
in Eq.~(\ref{eq:two_scale}).
\\
\\
\noindent
The extension to three dimensions is done using a standard tensor-product structure, 
where we get a basis of $(k+1)^3$ pure scaling functions on each cubic cell 
\begin{equation}
    \label{eq:tensor_basis}
    \Phi^n_{ijk}(x,y,z) = \phi^n_i(x)\phi^n_j(y)\phi^n_k(z)
\end{equation}
and we note that from the definition of the wavelet space in Eq.~(\ref{eq:wavelet-def}) 
we get a total of $(2^3-1)(k+1)^3$ wavelet functions on each cell (all combinations in 
Eq.~(\ref{eq:tensor_basis}) that contain at least one wavelet function), and it becomes 
apparent that the treatment of more than three dimensions is problematic as the size of 
the basis grows exponentially in the dimension.


\pagebreak
