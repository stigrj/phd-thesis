\chapter{Introduction}

\section{Numerical analysis and physics in the real world}
Simplifications and idealization for physics students.
Many physics students over the years have been struggling to find real world application 
of beeing able to calculate a cannon ball's trajectory ,

that the bird flying above your head, the violent roar of the water fall, the waves on the oscean, even the
movement of the moon and the sun across the sky, are all subject to the same fundamental laws of physics.

even if teh mathematical description of such real-world phenomena is way too complicated ... 
it is not difficult to imagine each and every water droplet as a tiny cannon ball 
floating through empty space, only slightly influenced by all its fellow water droplets, together
making up the turbulent, unpredictable ...of a water fall.

The bridge between ... is made through numerical analysis, which involves the translation of the
mathematical equations of the physical laws into the language of the digital computer.
Most modern applied sciences relies hevily upon numerical analysis and simulations, either for performing
numerically intensive calculations or for analyzing large amounds of data.

or the 
The true power with physics, of course, lies in its generality. 
Numerical simulations and modern applied science.
Real-world problems too complicated.
Increasing computational power. Moore's law. Parallel computers.

\section{Theoretical chemistry}
Chemistry is a branch science studying the composition, structure, properties and change of matter.
Chemistry as lab/cataloguing science before the itroduction of QM.
Through QM the complete physical theory became available, but although
the exact problem is decievingly simply stated for an arbitrary system through the Schr\"{o}dinger equation
\begin{equation}
    \hat{H}\Wavefunction = E\Wavefunction
\end{equation}
its solution for many-body problems is quite the opposite. In fact, whenever the system contains more than 
two particles the problem \emph{cannot} be solved (at least not in the usual sense in terms of the standard 
elementary functions of calculus).




Conventional KS/HF calculations routinely used.
Well known chemical notion of electron orbital, which for simple systems 
\begin{equation}
    \rho(\bs{r}) = \sum_i^N |\orbital_i(\bs{r})|^2
\end{equation}
orbital equations
\begin{equation}
    \hat{F}\orbital_i(\bs{r}) = \epsilon_i\orbital_i(\bs{r})
\end{equation}
While the solution of this set of $N$ coupled, non-linear, three-dimensional partial
differential equations is still a formidable computational task, the complexity of the 
full $3N$-dimensional Schr\"{o}dinger equation is sufficiently reduced for the numerical 
solution to be feasable for systems with a remarkable number of particles.

This has been made possible by combining a great deal of chemical intuition into the 
development of computational methods. In particular, the introduction of the atomic orbital 
basis in the form of atom-centered Gaussians can be attributed most of the success of modern 
computational chemistry, by providing efficient and compact representations with a consistent 
cancellation of errors. 

However, although the Gaussian basis is ideal for obtaining qualitative numbers fast, it 
struggles when high precision is required. Moreover, as the Gaussian functions extend 
throughout the entire system, it is difficult to reduce the problem into truly independent 
tasks that can be easily distributed among several computers and executed simultaneously.

The alternative to the elegant, compact representations using a carefully chosen, preoptimized 
atomic orbital basis, would be a brute force numerical solution using real-space representations 
in terms of numerical grids or finite elements. Such an approach would yield robust, unbiased
results that do not rely on cancellation of errors (but neither would it benefit from it).

It is a well-known fact that the electronic density in molecular systems is rapidly varying 
in the vicinity of the atomic nuclei, and a usual problem with real-space methods is that an 
accurate treatment of the system requires high resolution of grid points in the nuclear regions. 
Keeping this high resolution uniformly througout space would yield unnecessary high accuracy in 
the interatomic regions, thus the real-space treatment of molecular systems is demanding a 
\emph{multiresolution} framework in order to achieve numerical efficiency.

\section{Multiwavelets}
As the theory of wavelets is vast and can be considered a rather advanced topic
of applied mathematics, it remains unfamiliar to most chemists. However, 
Alpert's\cite{Alpert} construction of \emph{multiwavelets} is actually
rather simple. Starting with a small set of orthogonal polynomials 
$\lbrace\scaling_j\rbrace_{j=0}^k$ of order $\leq k$ on the unit interval, we 
attempt to represent a given function. If this basis turns out to be too crude 
to accurately describe the function, we can increase the flexibility by adding
higher order polynomials (thus increasing the polynomial order $k$), and we
approach a complete basis (and an exact representation) as $k\rightarrow\infty$.

Alpert shows that there is a second way to approach completeness in this basis.
Instead of increasing the polynomial order, we split the interval and double 
the number of basis functions by dilating and translating the original basis 
to both subintervals
\begin{equation}
    \scaling_{j,l}^1(x) = 2^{1/2}\scaling_j(2x - l), \qquad l = 0,1
\end{equation}
The polynomials are defined only on their respective intervals, so basis functions 
at different translations are by construction orthogonal. The splitting procedure 
can be continued until we have reached a scale $n$ where we are satisfied with the 
accuracy of the representation. At this level of refinement the unit interval has 
been split into $2^n$ intervals, each of size $2^{-n}$ containing a dilated and 
translated version of the original $k$-order basis
\begin{equation}
    \scaling_{j,l}^n(x) = 2^{n/2}\scaling_j(2^nx - l), \qquad l = 0,\dots,2^n-1
\end{equation}
This basis can be used to represent any square integrable function to any finite 
accuracy by adjusting the polynomial order $k$ and/or the level of refinement $n$. 

The construction in three dimensions is similar, where at refinement level $n$
the unit cube has been uniformly divided into $2^{3n}$ subcubes, each containing
a polynomial basis of $(k+1)^3$ functions constructed by tensor products of the
one-dimensional basis functions
\begin{equation}
    \scalingnd^n_{ijk}(x,y,z) = \scaling^n_i(x)\scaling^n_j(y)\scaling^n_k(z)
\end{equation}
The main advantage of multiwavelets over the similar finite element bases is the
possibility of constructing non-uniform grids, and thus focusing the computational
efforts into the problematic nuclear region. Moreover, the grid construction can 
be completely automated while the accuracy of the representation is guaranteed.

Although similar constructions were already familiar through the multigrid approaches 
within the finite element community, these methods suffered from a lack of mathematical
rigour and generality, with complicated problem-specific algorithms. Alpert's construction, 
on the other hand, was founded upon the well established, powerful theory of wavelets, 
making the basis applicable to a wide variety of physical problems and operators,
yielding sparse representations and fast algorithms.

\section{Organization of the thesis}
The multiwavelet basis is described in detail within the framework of multiresolution 
analysis in chapter \ref{chap:math}, and the practical implementation of this formalism
into a working computer code is presented in chapter \ref{chap:implementation}. In 
particular, we describe the mathematical operations necessary to solve the equations 
appearing in the self-consistent field methods of quantum chemistry. An introduction
to these methods are given in chapter \ref{chap:chemistry}, together with algorithms 
for their numerical solution.

Included in this thesis are also three papers submitted for publication, that can be
considered linked to each of the three main chapters. The first paper involves the
construction of the multiwavelet basis and is an attempt to reduce the memory 
requirements of the method by decreasing the polynomial order $k$ of the basis as 
the level of refinement $n$ is increased. 

The second paper describes the implementation of the code with particular focus on its
application to parallel computer architectures. The performance of the code (numerical
accuracy, linear scaling of computational time with respect to system size, and parallel
efficiency) is demonstrated on realistic molecular systems of up to 600 atoms.

The topic of the third paper is the solution of the self-consistent field problem
in quantum chemistry. General algorithms som are presented for the iterative solution



