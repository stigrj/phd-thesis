\chapter{Chemistry}
In this chapter we present the equations that govern chemical systems, in particular
the electronic stucture of atoms and molecules. At the moleculear length scale, nature
is most accurately described by the theory of quantum mechanics, where the central
problem is the solution of the non-relativistic Schr\"{o}dinger equation, the 
multi-dimensional partial differential equation that determine the wavefunction 
\Wavefunction that completely describes the quantum system.

\section{The molecular Schr\"{o}dinger equation}\label{sec:schrodinger}
\begin{equation}
    \hat{H}\Wavefunction = E\Wavefunction
\end{equation}
\begin{equation}
    \hat{H} = \hat{T}_{nuc} + \hat{H}_{el}
\end{equation}
\begin{align}
    \Wavefunction &= \wavefunction_{nuc}\wavefunction_{el}\\
    E &= E_{nuc} + E_{el}
\end{align}
\begin{equation}
    \hat{T}_{nuc} = -\sum_I\frac{1}{2m_I} \nabla_I^2
\end{equation}
\begin{equation}
    \hat{H}_{el} = -\sum_i \frac{1}{2}\nabla_i^2 
	+ \sum_{i>j} \frac{1}{|\bs{r}_j-\bs{r}_i|}
	- \sum_{I,i} \frac{Z_I}{|\bs{r}_i-\bs{R_I}|}
        + \sum_{I>J} \frac{Z_IZ_J}{|\bs{R}_I-\bs{R_J}|}
\end{equation}
\begin{equation}
    \hat{H}_{el}\wavefunction_{el} = E_{el}\wavefunction_{el}
\end{equation}
\begin{equation}
    \wavefunction(\bs{x}_1,\bs{x}_2,\dots,\bs{x}_N) = -
    \wavefunction(\bs{x}_2,\bs{x}_1,\dots,\bs{x}_N)
\end{equation}
\begin{equation}
    \label{eq:var_princ}
    \frac{\langle\tilde{\Wavefunction}|\hat{H}|\tilde{\Wavefunction}\rangle}
    {\langle\tilde{\Wavefunction}|\tilde{\Wavefunction}\rangle}
    \geq
    \frac{\langle\Wavefunction_0|\hat{H}|\Wavefunction_0\rangle}
    {\langle\Wavefunction_0|\Wavefunction_0\rangle}
\end{equation}

\section{Hartree-Fock Theory}\label{sec:HFT}
\begin{equation}
    \wavefunction(\bs{x}_1,\bs{x}_2,\dots,\bs{x}_N) = \sum_m c_m 
	\orbital_1^m(\bs{x}_1)
	\orbital_2^m(\bs{x}_2)\cdots
	\orbital_N^m(\bs{x}_N)
\end{equation}
\begin{equation}
    \begin{split}
    \wavefunction(\bs{x}_1,\bs{x}_2,\dots,\bs{x}_N) = 
    |\orbital_1\orbital_2\cdots\orbital_N\rangle \mydef \frac{1}{\sqrt{N}} 
    \left|
    \begin{array}{cccc}
	\orbital_1(\bs{x}_1)	& \orbital_1(\bs{x}_2)	& \cdots & \orbital_1(\bs{x}_N)\\
	\orbital_2(\bs{x}_1)	& \orbital_2(\bs{x}_2)	& \cdots & \orbital_2(\bs{x}_N)\\
	\vdots			& \vdots		& \ddots & \vdots\\
	\orbital_N(\bs{x}_1)	& \orbital_N(\bs{x}_2)	& \cdots & \orbital_N(\bs{x}_N)
    \end{array}
    \right|
    \end{split}
\end{equation}
\begin{equation}
    \hat{h}_0 = \sum_{I<J} \frac{Z_IZ_J}{|\bs{R}_I-\bs{R}_J|}\bigg)
\end{equation}
\begin{equation}
    \hat{h}_i = \bigg(-\frac{1}{2}\nabla_i^2
	-\sum_I \frac{Z_I}{|\bs{r}_i-\bs{R}_I|}\bigg)
\end{equation}
\begin{equation}
    \hat{g}_{ij} = \frac{1}{|\bs{r}_j-\bs{r}_i|}
\end{equation}
\begin{equation}
    \hat{H} = \hat{h}_0 + \sum_i \hat{h}_i + \sum_{i<j} \hat{g}_{ij}
\end{equation}
The energy of a Slater determinant wave function is evaluated as the expectation value
of the Hamiltonian, and is given by
\begin{align}
    E[\wavefunction] &=
    \langle\orbital_1\orbital_2\cdots\orbital_N|\hat{H}|
    \orbital_1\orbital_2\cdots\orbital_N\rangle\\
    &= \hat{h}_0 + \sum_{i=1}^N \langle \orbital_i |\hat{h}| \orbital_i \rangle +
    \frac{1}{2} \sum_{i=1}^N\sum_{j=1}^N \Big(
    \langle \orbital_i\orbital_j |\hat{g}| \orbital_i\orbital_j\rangle -
    \langle \orbital_i\orbital_j |\hat{g}| \orbital_j\orbital_i\rangle\Big)
\end{align}
where the one- and two-electron integrals are
\begin{align}
    \langle \orbital_i|\hat{O}|\orbital_j \rangle &= \int\orbital_i^{\ast}(\bs{x}_1)
    \hat{O}\orbital_j(\bs{x}_1) \ud \bs{x}_1\\
    \langle \orbital_i\orbital_j|\hat{O}|\orbital_k\orbital_l \rangle &= \int \int 
    \orbital_i^{\ast}(\bs{x}_1)\orbital_j^{\ast}(\bs{x}_2)
    \hat{O}\orbital_k(\bs{x}_1)\orbital_l(\bs{x}_2) \ud \bs{x}_1 \ud \bs{x}_2
\end{align}
Introducing the Coulomb $\hat{J}=\sum_i\hat{J}_i$ and exchange $\hat{K}=\sum_i\hat{K}_i$
operators, which are defined through their action on an arbitrary function $f$ 
\begin{align}
    \hat{J}_j f(\bs{x}_1) &= \bigg(\int \frac{\orbital_j^{\ast}(\bs{x}_2)\orbital_j(\bs{x}_2)}
	{|\bs{x}_2-\bs{x}_1|}\ud\bs{x}_2\bigg) f(\bs{x}_1)\\
    \hat{K}_j f(\bs{x}_1) &= \bigg(\int \frac{\orbital_j^{\ast}(\bs{x}_2)f(\bs{x}_2)}
	{|\bs{x}_2-\bs{x}_1|}\ud\bs{x}_2\bigg) \orbital_j(\bs{x}_1)
\end{align}
we arrive at the expression for the total electronic energy of a Slater determinant
\begin{equation}
    E[\wavefunction] = \hat{h}_0 +
    \sum_{i=1}^N \langle \orbital_i |\hat{h}| \orbital_i \rangle +
    \frac{1}{2} \sum_{i=1}^N 
    \langle \orbital_i |\hat{J}-\hat{K}| \orbital_i\rangle
\end{equation}
The Hartree-Fock energy $E_{HF}$ is obtained by minimizing the energy with respect 
to orbital variations
\begin{equation}
    E_{HF} = \mymin{\wavefunction}\ E[\wavefunction]
\end{equation}
following the variational principle of Eq.~(\ref{eq:var_princ}), under the constraint 
that the orbitals remain orthonormal $\langle\orbital_i|\orbital_j\rangle = \delta_{ij}$.

\section{Density Functional Theory} \label{sec:DFT}
Energy expression\\
Hohenberg-Kohn theorems\\

\subsection{Density functionals}
Thomas-Fermi-Dirac\\
von Wiesacker\\
Borgoo\\

\subsection{Kohn-Sham theory}
Energy expression\\

In the absence of any externally applied fields, there are three contributions 
to the effective potential in Kohn-Sham theory.
\begin{equation}
    \label{eq:eff_pot}
    V_{eff} = V_{nuc} + V_{coul} + V_{xc}
\end{equation}
In addition to the nuclear potential $V_{nuc}$ presented above, there is the 
classical Coulomb repulsion between the electrons $V_{coul}$ as well as the 
non-classical exchange-correlation potential $V_{xc}$. The Coulomb potential 
is described in terms of the Poisson equation, which is expressed in integral 
form using the Green's function of Eq.~(\ref{eq:greens_kernel})
\begin{equation}
    \label{eq:coulomb}
    V_{coul}(r) = \int P(r-r')\rho(r')\ud r'
\end{equation}
A highly efficient (linear scaling) and parallelized implementation of the 
solution to the Poisson equation in the multiwavelet basis has been presented 
in a previuos study\cite{mwpar}. 

\subsection{Exchange-Correlation potentials}
The exchange-correlation potential is formally expressed in terms of the 
functional derivative of the exchange-correlation energy
\begin{equation}
    V_{xc} 	= \frac{\delta E_{xc}[\rho]}{\delta \rho} 
		= \frac{\delta}{\delta \rho} \int F_{xc}(\rho,|\nabla\rho|,\nabla^2\rho,\dots) \ud \rho
\end{equation}
which in the local density approximation (LDA) reduces to
\begin{equation}
    V_{xc} = \frac{\partial F_{xc}(\rho)}{\partial \rho}
\end{equation}

\subsection{Open shell systems} \label{sec:open_shell_KS}
The extension to open shell systems is straightforward. In the local spin 
density approximation (LSDA) we separate the orbitals into $\alpha$ and 
$\beta$ spins, with corresponding spin densities 
\begin{equation}
    \rho^{\sigma} (r) = \sum_{i=1}^{N_{\sigma}} |\phi_i^{\sigma}(r)|^2 \qquad \sigma = \alpha, \beta
\end{equation} 
where the different spin orbitals experience different effective potentials
\begin{equation}
    V_{eff}^{\sigma} = V_{nuc} + V_{coul} + V_{xc}^{\sigma} \qquad \sigma = \alpha, \beta
\end{equation}
All spin effects are included in the exchange-correlation potential, so the 
nuclear and Coulomb potentials are the same as in Eq.~(\ref{eq:eff_pot}), 
where the latter is obtained from the full ($\alpha + \beta$) electron density.



\section{Integral formulation}

\subsection{Single-orbital systems}
\subsection{Many-orbital systems}
\subsection{Orbital-free DFT}


\subsection{Kohn-Sham equations}
For a closed shell system in Kohn-sham DFT the ground state electron density is given 
in terms of the occupied Kohn-Sham orbitals (open shell systems are 
treated in section \ref{sec:open_shell_KS})
\begin{equation}
    \rho(r) = 2 \sum_{i=1}^N |\varphi_i(r)|^2
\end{equation}
where the occupied orbitals are the $N$ lowest eigenfunctions of the 
Kohn-Sham operator
\begin{equation}
    \label{eq:KS-diff}
    \big[-\frac{1}{2}\nabla^2 + V_{eff}(r)\big] \varphi_i(r) = \epsilon_i \varphi_i(r)
\end{equation}
where the effective potential (defined below) depends implicitly on the 
orbitals through the density. This means that we have a non-linear problem with $N$ coupled 
equations that must be solved self-consistently.

\subsection{Integral formulation}\label{sec:1_orb_int}
Although the differential form in Eq.~(\ref{eq:diff_schrodinger}) is most 
commonly used, it is not very well suited for solution in the multiwavelet 
basis \cite{Harrison}, and differential operators (in particular higher 
order operators) should be avoided in order to maintain high accuracy. 
Kalos \cite{Kalos} showed in 1962 that the Schr\"{o}dinger equation can 
be formulated as an integral equation, using the Helmholtz Green's functions 
of Eq.~(\ref{eq:greens_kernel}), switching to atomic units
\begin{align}
    \big[-\frac{1}{2}\nabla^2 + V(r)\big]\varphi(r) &= E \varphi(r)\\
    \big[-\nabla^2 - 2E\big]\varphi(r) &= -2V(r) \varphi(r)\\
    \varphi(r) &= -2\int H_{\mu}(r-r')V(r') \varphi(r') \ud r'\\
    \label{eq:int_schrodinger}
    \varphi &= -2 \hat{H}\big[V\varphi\big]
\end{align}
with $\mu = \sqrt{-2E}$. Eq.~(\ref{eq:int_schrodinger}) defines a fixed-point 
problem, and iterative solution techniques for the determination of the 
wavefunction $\varphi$ and its ''energy'' $\mu$ are presented in section 
\ref{sec:iterative_solution}.


\subsection{Integral formulation}
As for the one-electron problem, we want to rewrite the differential 
equations in Eq.~(\ref{eq:KS-diff}) into integral form in order to be 
solved using the multiwavelet basis. This can be done straightforwardly
by the procedure of section \ref{sec:1_orb_int} for each orbital separately, 
\begin{equation}
    \label{eq:KS-int}
    \varphi_i(r) = -2\int H_{\mu_i}(r-r')V_{eff}(r') \varphi_i(r') \ud r'
\end{equation}
and the equations are again solved iteratively
\begin{equation}
    \label{eq:KS-iter}
    \tilde{\varphi}^{n+1}_i = -2 \hat{H}^n\big[V^n_{eff}\varphi^n_i\big]
\end{equation}
although care must be taken in order to keep orthonormality between the 
orbitals. A simple iteration of all $N$ orbitals separately will not 
conserve the canonical character of the input orbitals, and all orbitals 
will eventually collapse into the lowest energy eigenfunction, unless 
orthogonality is explicitly enforced. There are different ways in which 
one can keep orthonormality, and these are discussed below. Note also the
iteration label on the effective potential, as this depends on the orbitals 
and must be updated in each iteration.

\section{Basis sets in computational chemistry}

