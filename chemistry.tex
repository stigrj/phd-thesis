\chapter{Electronic structure theory}
In this chapter we present the equations that govern chemical systems, in particular
the electronic stucture of atoms and molecules. At the molecular length scale, nature
is most accurately described by the theory of quantum mechanics, where the central
problem is the solution of the non-relativistic Schr\"{o}dinger equation.

Being that this problem cannot be solved exactly by any analytical method whenever
the system contains more than two particles, much of the work in the field of quantum
chemistry has been concerned with developing accurate and efficient approximations,
a work that has been given invaluable support by the remarkable increase in 
computational power that has taken place since the advent of the electronic computer 
more than halv a century ago.

This chapter will give an introduction to the Self-Consistent Field (SCF)
approximations that are commonly employed in computational chemistry. We will start with a
traditional presentation of the orbital based methods of Hartree-Fock and Kohn-Sham 
Density Functional Theory, where the aim of the chapter
is to rewrite the equations into their less familiar integral formulation,
a formulation much better suited for the treatment in the multiwavelet framework 
presented in chapter \ref{chap:math}. Algorithms for how to solve these equations
using the mathematical tools as implemented in chapter \ref{chap:implementation} 
is the topic of publication 3.

\section{The molecular Schr\"{o}dinger equation}\label{sec:schrodinger}
The physical state of a quantum system influenced by potentials that do not change
with time is described by the time-independent Schr\"{o}dinger equation 
\begin{equation}
    \label{eq:mol_schr}
    \hat{H}\Wavefunction = E\Wavefunction
\end{equation}
where the Hamiltonian $\hat{H}$ is the operator for the total energy $E$ of the system.
The wave function \Wavefunction\ is an eigenfunction of the Hamiltonian operator, 
and is a multi-dimensional (in general complex-valued) function that depends on the 
degrees of freedom of the system, e.i. the position $\bs{r}$ and spin $s$ of all $N$ 
particles, and we have $\Wavefunction=\Wavefunction(\bs{x}_1,\bs{x}_2,\dots,\bs{x}_N)$, 
where $\bs{x}_i=(\bs{r}_i,s_i)$ denotes the position and spin of the $i$-th particle. 
There are in general infinitely many eigenfunctions for a given Hamiltonian operator,
each corresponding to a possible state of the system. 

The wave function contains all the information that can possibly be extracted from the 
physical system. For each physical observable $\Omega$ there is an associated mathematical 
operator $\hat{\Omega}$, such that the expected value of an experimental measurement is given by
\begin{equation}
    \langle \hat{\Omega}\rangle = \frac{\langle\Wavefunction|\hat{\Omega}|\Wavefunction\rangle}
    {\langle\Wavefunction|\Wavefunction\rangle}
\end{equation}
This means that the fundamental problem in quantum chemistry is to obtain the molecular
wave function by solving the Schr\"{o}dinger equation (\ref{eq:mol_schr}). For a molecule,
the Hamiltonian contains kinetic $\hat{T}$ and potential $\hat{V}$ energy of the 
electrons and nuclei that make up the system
\begin{equation}
    \hat{H} = \hat{T}_{nuc} + \hat{T}_{el} + \hat{V}_{n-n} + \hat{V}_{e-e} + \hat{V}_{n-e}
\end{equation}
Analytical solutions exists only for the one- and two-particle problems, and approximations are
inevitable if we want to be able to treat more interresting chemical systems. 

The first approximation for molecular systems is almost exclusively the Born-Oppenheimer 
approximation\cite{BornOppenheimer},
in which we consider the nuclei to be fixed in space, so that the electrons move in a static
nuclear potential. The motivation for this approximation is that the nuclei are much heavier
than the electrons, and hence move much slower, so that at the electronic time scale, the
nuclei are percieved as classical particles frozen in space. This means that we can disregard the
instantaneous correlation between the electrons and the nuclei, and we can separate the nuclear
kinetic energy from an electronic Hamiltonian
\begin{align}
    \hat{H} &= \hat{T}_{nuc} + \hat{H}_{el}\\
    \hat{H}_{el} &= \hat{T}_{el} + \hat{V}_{n-e} + \hat{V}_{e-e} + \hat{V}_{n-n}
\end{align}
In atomic units\footnote{$e=m_e=\hbar=4\pi\epsilon_0=1$}, using uppercase indices for the
nuclei and lowercase indices for the electrons, we have the electron kinetic energy
\begin{equation}
    \hat{T}_{el} = -\sum_i \frac{1}{2}\nabla_i^2
\end{equation}
the electron-nuclear attraction
\begin{equation}
    \hat{V}_{n-e} = -\sum_{i,I} \frac{Z_I}{\|\bs{r}_i-\bs{R_I}\|}
\end{equation}
the electron-electron repulsion
\begin{equation}
    \hat{V}_{e-e} = \sum_{i>j} \frac{1}{\|\bs{r}_j-\bs{r}_i\|}
\end{equation}
and finally the nuclear-nuclear repulsion
\begin{equation}
    \hat{V}_{n-n} = \sum_{I>J} \frac{Z_IZ_J}{\|\bs{R}_I-\bs{R_J}\|}
\end{equation}
Within the Born-Oppenheimer approximation, the last term is a simple additive constant 
and is usually left out when solving the electronic problem
\begin{equation}
    \label{eq:el_schr}
    \hat{H}_{el}\wavefunction_{el} = E_{el}\wavefunction_{el}
\end{equation}
At the nuclear time scale, the electrons are percieved as a diffuse charge density that is
able to respond instantaneously to the movement of the nuclei, and molecular rotations and 
vibrations are described by the nuclear wave function which is influenced by this dynamic
electron density. In the following, however, we are concerned exclusively with the calculation 
of the electronic wave function through Eq.~(\ref{eq:el_schr}), where the $el$ subscript 
henceforth will be dropped.

The particular state $\wavefunction_0$ with the lowest energy $E_0$ is called 
the electronic ground state of the system and serves special attention in quantum chemistry.
The reason for this is that for most chemical systems the ground state is the only state
significantly populated under normal laboratory conditions, and hence, most chemical 
phenomena can be explained in terms of properties of the electronic ground state. The way to
calculate the ground state is usually to exploit the variational principle, which states that
for a given Hamiltonian $H$ with true ground state $\wavefunction_0$, we have, for an arbitrary
trial wave function $\tilde{\wavefunction}$
\begin{equation}
    \label{eq:var_princ}
    \frac{\langle\tilde{\wavefunction}|\hat{H}|\tilde{\wavefunction}\rangle}
    {\langle\tilde{\wavefunction}|\tilde{\wavefunction}\rangle}
    \geq
    \frac{\langle\wavefunction_0|\hat{H}|\wavefunction_0\rangle}
    {\langle\wavefunction_0|\wavefunction_0\rangle}
\end{equation}
which means that finding the ground state can be regarded as a minimization problem, where
the trial wave function is varied to the point where the corresponding energy is minimized.

\section{Hartree-Fock Theory}\label{sec:HFT}
The most apparent complication in developing approximate methods for the solution of the 
electronic Sch\"{o}dinger equation is perhaps the high dimensionality of the problem. For
a system containing $N$ electrons, the wave function is a $3N$-dimensional scalar function 
(disregarding spin). The common way to approach such high-dimensional problems is by
approximating the full $d$-dimensional function in terms of products of functions of lower 
dimensionality. In chemistry it is convenient to use one-particle function, called spin-orbitals,
which depend on the coordinates of a single electron
\begin{equation}
    \label{eq:orb_exp}
    \wavefunction(\bs{x}_1,\bs{x}_2,\dots,\bs{x}_N) = \sum_m c_m 
	\orbital_1^m(\bs{x}_1)
	\orbital_2^m(\bs{x}_2)\cdots
	\orbital_N^m(\bs{x}_N)
\end{equation}
Unfortunately, the convergence of such expansions is not very good, and a large number
of terms is usually required in order to obtain high accuracy. One way of improving
the convergence is to include two-particle functions in the expansion. Such approaches,
known as \emph{explicitly correlated methods}\cite{klopper,valeev}, will not be discussed
further at this point, and in the following we will use wave functions constructed as a 
single Slater determinant.

\subsection{Slater determinant}
Being fermionic, the electronic wave function needs to be anti-symmetric with respect to 
the exchange of two particles
\begin{equation}
    \wavefunction(\bs{x}_1,\bs{x}_2,\bs{x}_3,\dots,\bs{x}_N) = -
    \wavefunction(\bs{x}_2,\bs{x}_1,\bs{x}_3\dots,\bs{x}_N)
\end{equation}
This condition is known as the Pauli exclusion principle\cite{Pauli_princ}, which has the 
consequence that each fermionic state can only be occupied by one particle. The simplest way
of constructing a wave function that fulfils the anti-symmetry requirement using one-particle
spin-orbitals is the Slater determinant\cite{slater}
\begin{equation}
    \begin{split}
    \wavefunction = |\orbital_1\orbital_2\cdots\orbital_N\rangle \mydef \frac{1}{\sqrt{N!}} 
    \left|
    \begin{array}{cccc}
	\orbital_1(\bs{x}_1)	& \orbital_1(\bs{x}_2)	& \cdots & \orbital_1(\bs{x}_N)\\
	\orbital_2(\bs{x}_1)	& \orbital_2(\bs{x}_2)	& \cdots & \orbital_2(\bs{x}_N)\\
	\vdots			& \vdots		& \ddots & \vdots\\
	\orbital_N(\bs{x}_1)	& \orbital_N(\bs{x}_2)	& \cdots & \orbital_N(\bs{x}_N)
    \end{array}
    \right|
    \end{split}
\end{equation}
where the spin-orbitals $\orbital_i(\bs{x})$ are orthonormal and can be expressed as a product 
of a three-dimensional spatial part and a spin part. The energy of such a wave function is 
evaluated as the expectation value of the Hamiltonian
\begin{align}
    E[\wavefunction] 
	&= \langle\orbital_1\orbital_2\cdots\orbital_N|\hat{H}|
	\orbital_1\orbital_2\cdots\orbital_N\rangle\\
	&= \sum_{i=1}^N \langle \orbital_i |\hat{h}| \orbital_i \rangle +
	\frac{1}{2} \sum_{i=1}^N \sum_{j=1}^N
	\langle \orbital_i |\hat{J}_j-\hat{K}_j| \orbital_i\rangle
\end{align}
where we have defined the one-electron operator
\begin{equation}
    \hat{h}\orbital_i(\bs{x}) = \bigg(-\frac{1}{2}\nabla^2 - \sum_I\frac{Z_I}{\|\bs{r}-\bs{R}_I\|}\bigg)
	\orbital_i(\bs{x})
\end{equation}
as well as the Coulomb $\hat{J}_j$ and exchange $\hat{K}_j$ operators
\begin{align}
    \hat{J}_j \orbital_i(\bs{x}) &= \bigg(\int \frac{\orbital_j^{\ast}(\bs{x}')\orbital_j(\bs{x}')}
	{\|\bs{r}-\bs{r}'\|}\ud\bs{x}'\bigg) \orbital_i(\bs{x})\\
    \hat{K}_j \orbital_i(\bs{x}) &= \bigg(\int \frac{\orbital_j^{\ast}(\bs{x}')\orbital_i(\bs{x}')}
	{\|\bs{r}-\bs{r}'\|}\ud\bs{x}'\bigg) \orbital_j(\bs{x})
\end{align}
where it is important to note that the integration is over space \emph{and} spin coordinates,
which means that the exchange operator is zero if the spin of orbitals $i$ and $j$ differ. 
The Coulomb operator, on the other hand, in non-vanishing for all spin-orbitals.

\subsection{The Hartree-Fock equations}
The best approximation to the ground state in terms of a \emph{single} Slater determinant is 
called the Hartree-Fock wave function, and is obtained by minimizing the energy with respect 
to orbital variations
\begin{equation}
    E_0 = \mymin{\wavefunction}\ E[\wavefunction]
\end{equation}
following the variational principle of Eq.~(\ref{eq:var_princ}). By imposing the constraint 
that the orbitals remain orthonormal $\langle\orbital_i|\orbital_j\rangle = \delta_{ij}$
by means of Lagrange multipliers, we arrive at the Hartree-Fock equations
\begin{equation}
    \hat{F}\orbital_i(\bs{x}) = \epsilon_i \orbital_i(\bs{x})
\end{equation}
where the Fock operator is given as
\begin{equation}
    \hat{F} = \hat{h} + \sum_j^N \Big(\hat{J}_j - \hat{K}_j\Big)
\end{equation}
The Hartree-Fock wave function is then obtained as the Slater determinant constructed by 
the $N$ lowest energy eigenfunctions $\orbital_i$ of the Fock operator.

Some of the terms included in the Fock operator can be expressed as multiplicative
potentials instead of operators. The core Hamiltonian $\hat{h}$ includes the scalar 
electrostatic potential arising from the nuclear charges
\begin{equation}
    v_{nuc}(\bs{r}) = \sum_{I} \frac{Z_I}{\|\bs{r} - \bs{R}_I\|}
\end{equation}
and the sum of the Coulomb operators is nothing but the scalar electrostatic potential
arising from all electrons of the system
\begin{equation}
    v_{el}(\bs{r}) = \sum_j^N \hat{J}_j = \sum_j^N \int \frac{|\orbital_j(\bs{x}')|^2}
	{\|\bs{r} - \bs{r}'\|} \ud \bs{x}'
\end{equation}
If we now collect the sum of exchange operators into a single operator
\begin{equation}
    \hat{K}\orbital_i(\bs{x}) = \sum_j^N \hat{K}_j \orbital_i(\bs{x}) 
	= \sum_j^N \orbital_j (\bs{x}) \int \frac{\orbital_j^{\ast}(\bs{x}')\orbital_i(\bs{x}')}
	    {\|\bs{r} - \bs{r}'\|} \ud \bs{x}'
\end{equation}
we can write the Hartree-Fock equations as
\begin{equation}
    \label{eq:HF_equations}
    \Big[-\frac{1}{2}\nabla^2 + v_{nuc}(\bs{r}) + v_{el}(\bs{r}) - \hat{K}\Big]\orbital_i(\bs{x}) 
	= \epsilon_i \orbital_i(\bs{x})
\end{equation}
As both the electronic potential $v_{el}$ and the exchange operator $\hat{K}$ depend on the set of
occupied orbitals, we have a set of coupled non-linear differential equations that needs to
be solved iteratively until we have a self-consistent solution. The main deficiancy of such
a Self-Consistent Field (SCF) approximation is that each electron only interacts with the average field
created by the other electrons. While this is a good approximation for the electron's interaction
with the slow moving nuclei, the instantaneous correlation is much more important between two electrons,
and needs to be taken into account if high precision is required. There exists several post-Hartree-Fock
methods that model this missing correlation energy, including configuration interaction and coupled-cluster
theory, but these will not be discussed in this thesis.

\section{Density Functional Theory} \label{sec:DFT}
\begin{equation}
    \rho(\bs{r}_1) = N \int |\wavefunction(\bs{x}_1,\bs{x}_2,\dots,\bs{x}_N)|^2 
	\ud s_1 \ud \bs{x}_2 \cdots \ud \bs{x}_N
\end{equation}
\begin{equation}
    E[\rho] = T[\rho] + V_{ne}[\rho] + V_{ee}[\rho]
\end{equation}
\begin{align}
    V_{ne}[\rho] &= \int \rho(\bs{r})v_{nuc}(\bs{r}) \ud \bs{r}\\
    F[\rho] &= T[\rho] + V_{ee}[\rho]
\end{align}
\begin{equation}
    E_0 = \mymin{\rho}\ E[\rho]
\end{equation}
with the constraints that the density is everywhere positive and integrates to the number
of electrons.

\subsection{Kohn-Sham equations}
\begin{equation}
    \rho(\bs{r}) = 2 \sum_i^{N/2} | \orbital_i(\bs{r})|^2
\end{equation}
\begin{equation}
    F[\rho] = T_s[\rho] + J[\rho] + E_{xc}[\rho]
\end{equation}
\begin{equation}
    T_s[\rho] = \sum_i^{N/2} \langle \orbital_i | -\frac{1}{2}\nabla^2 | \orbital_i \rangle
\end{equation}
\begin{equation}
    J[\rho] = \frac{1}{2} \int \rho(\bs{r})v_{el}(\bs{r}) \ud \bs{r}
\end{equation}
\begin{equation}
    E_{xc}[\rho] = T[\rho] - T_s[\rho] + V_{ee}[\rho] - J[\rho]
\end{equation}
\begin{equation}
    \label{eq:KS-energy}
    E[\rho] = T_s[\rho] + V_{en}[\rho] + J[\rho] + E_{xc}[\rho]
\end{equation}
leads to the Euler equation
\begin{equation}
    \mu = \frac{\delta T_s[\rho]}{\delta \rho(\bs{r})} + v_{eff}(\bs{r})
\end{equation}
where the chemical potential $\mu$ is a Lagrange multiplier that fixes the number of 
electrons and the effective potential is given by
\begin{align}
    v_{eff}(\bs{r}) 
	&= \frac{\delta V_{en}[\rho]}{\delta \rho(\bs{r})}
	+ \frac{\delta J[\rho]}{\delta \rho(\bs{r})}
	+ \frac{\delta E_{xc}[\rho]}{\delta \rho(\bs{r})}\\
	&= v_{nuc}(\bs{r}) + v_{el}(\bs{r}) + v_{xc}(\bs{r})
\end{align}
The Fock operator for the system of non-interacting electrons influenced by an effective
potential is given simply as
\begin{equation}
    \hat{F}_{KS} = -\sum_i^{N/2} \frac{1}{2}\nabla_i^2 + \sum_i^{N/2} v_{eff}(\bs{r}_i)
\end{equation}
and is also called the Kohn-Sham operator. As there are no couplings between the electrons, 
this operator is separable, and the exact wave function is given by a single determinant 
constructed by the $N/2$ lowest energy eigenfunctions of the Kohn-Sham operator
\begin{equation}
    \Big[-\frac{1}{2}\nabla^2 + v_{nuc}(\bs{r}) + v_{el}(\bs{r}) + v_{xc}(\bs{r})\Big] 
	\orbital_i(\bs{r}) = \epsilon_i \orbital_i(\bs{r})
\end{equation}
where each orbital appears in the determinant with $\alpha$ and $\beta$ spins.

\subsection{The Exchange-Correlation potential}
The exchange-correlation energy functional is given as an integral over an energy
density $F_{xc}$
\begin{equation}
    E_{xc}[\rho] = \int F_{xc}\ud \bs{r}
\end{equation}
In the local density approximation (LDA) the energy density is a function of the density 
alone $F_{xc}(\rho)$, in the generalized gradient approximation (GGA) it is a function of
the density and its gradient $F_{xc}(\rho, |\nabla\rho|)$, while in meta-GGA's, higher order 
derivatives are introduced $F_{xc}(\rho, |\nabla\rho|, \nabla^2\rho, \cdots)$. Hybrid 
functionals, such as the very popular B3LYP\cite{B3LYP}, are GGA's with a certain amount of
exact Hartree-Fock exchange, evaluated as in Eq.~(\ref{eq:HFX}) using Kohn-Sham orbitals.

The exchange-correlation potential was implicitly defined in Eq.~(\ref{eq:XXX}) as the
functional derivative of the exchange-correlation energy with respect to the density
\begin{equation}
    v_{xc} = \frac{\delta E_{xc}[\rho]}{\delta \rho} 
	= \frac{\delta}{\delta \rho} \int F_{xc} \ud \bs{r}
\end{equation}
which in LDA reduces to
\begin{equation}
    v_{xc}^{LDA} = \frac{\partial F_{xc}}{\partial \rho}
\end{equation}
while a second derivative of $F_{xc}$ is required for GGA's
\begin{equation}
    v_{xc}^{GGA} = \frac{\partial F_{xc}}{\partial \rho} - 
	\nabla\cdot\frac{\partial F_{xc}}{\partial\nabla\rho}
\end{equation}

\subsection{Spin-unrestricted Kohn-Sham}
The extension to spin-unrestricted and open-shell systems is straightforward. In this case the
$\alpha$ and $\beta$ electrons occupy different spatial orbitals, $\orbital^\alpha$ and 
$\orbital^\beta$, and we define the corresponding spin densities
\begin{equation}
    \rho^{\sigma}(\bs{r}) = 
	\sum_{i=1}^{N_{\sigma}} |\phi_i^{\sigma}(\bs{r})|^2, \qquad \sigma = \alpha, \beta
\end{equation}
All spin effects are included in the exchange-correlation potential, which in this case will 
depend on the spin densities and possibly their gradients, and the $\alpha$ and $\beta$ electrons 
will experience different effective potentials
\begin{equation}
    v_{eff}^{\sigma} (\bs{r}) = 
	v_{nuc}(\bs{r}) + v_{el}(\bs{r}) + v_{xc}^{\sigma}(\bs{r}), \qquad \sigma = \alpha, \beta
\end{equation}
The nuclear potential is the same as before, whereas the electronic potential is obtained
from the total electron density
\begin{equation}
    v_{el}(\bs{r}) = \int \frac{\rho^\alpha(\bs{r}') + \rho^\beta(\bs{r}')}
	{\|\bs{r}-\bs{r}'\|} \ud \bs{r}'
\end{equation}
This leads to different Kohn-Sham operators for the different spins
\begin{equation}
    \hat{F}_{KS}^\sigma = -\sum_i^{N_\sigma} \frac{1}{2}\nabla_i^2 + 
	\sum_i^{N_\sigma} v_{eff}^\sigma(\bs{r}_i), \qquad \sigma = \alpha, \beta
\end{equation}
and a single determinant wave function is constructed by the $N_\alpha$ and $N_\beta$ 
lowest energy eigenfunctions of the operators $\hat{F}_{KS}^\alpha$ and 
$\hat{F}_{KS}^\beta$, respectively.

\section{Basis sets in computational chemistry}

\section{Integral formulation}
Although the differential form in Eq.~(\ref{eq:diff_schrodinger}) is most 
commonly used, it is not very well suited for solution in the multiwavelet 
basis \cite{Harrison}, and differential operators (in particular higher 
order operators) should be avoided in order to maintain high accuracy. 
Kalos \cite{Kalos} showed in 1962 that the Schr\"{o}dinger equation can 
be formulated as an integral equation, using the Helmholtz Green's functions 

\begin{equation}
    g(\bs{r}) = \int K(\bs{r}-\bs{r}') f(\bs{r}') \ud\bs{r}
\end{equation}
using the Poisson and bound-state Helmholtz integral kernels
\begin{equation}
    P(\bs{r}-\bs{r}') = \frac{1}{4\pi\|\bs{r}-\bs{r}'\|} \qquad
    H^\mu(\bs{r}-\bs{r}') = \frac{e^{-\mu\|\bs{r}-\bs{r}'\|}}{4\pi\|\bs{r}-\bs{r}'\|}
\end{equation}

\subsection{One-electron systems}
\begin{align}
    \big[-\frac{1}{2}\nabla^2+v_{nuc}(\bs{r})\big]\wavefunction(\bs{r}) &= E \wavefunction(\bs{r})\\
    \big[-\nabla^2 - 2E\big]\wavefunction(\bs{r}) &= -2v_{nuc}(\bs{r}) \wavefunction(\bs{r})\\
    \wavefunction(\bs{r}) &= -2\int H^{\mu}(\bs{r}-\bs{r}')v_{nuc}(\bs{r}') 
	\wavefunction(\bs{r}') \ud \bs{r}'\\
    \label{eq:int_schrodinger}
    \wavefunction &= -2 \hat{G}_{\mu}\big[v_{nuc}\wavefunction\big]
\end{align}
with $\mu = \sqrt{-2E}$.


\subsection{Many-electron systems}

\begin{equation}
    F_{ij} = \langle \orbital_i | \hat{F} | \orbital_j \rangle
\end{equation}
\begin{equation}
    \hat{F}| i \rangle	
	= \Big[\sum_j |j \rangle\langle j|\Big] \hat{F}|i \rangle 
	= \sum_j F_{ji} |j \rangle
\end{equation}

\subsubsection{Kohn-Sham DFT}
\begin{align}
    \big[-\frac{1}{2}\nabla^2+v_{eff}(\bs{r})\big]\orbital_i(\bs{r}) 
	    &= \sum_j F_{ji} \orbital_j(\bs{r})\\
    \big[-\nabla^2 - 2\lambda\big]\orbital_i(\bs{r}) &= -2\Big[v_{eff}(\bs{r}) \orbital_i(\bs{r})
	    + \sum_j \big(\lambda\delta_{ij} - F_{ji}\big) \orbital_j(\bs{r})\Big]\\
    \orbital_i &= -2\hat{G}_\mu\Big[v_{eff} \orbital_i + 
	\sum_j \big(\lambda\delta_{ij} - F_{ji}\big) \orbital_j\Big]
\end{align}
where $\mu = \sqrt{-2\lambda}$. This general expression can be simplified in many ways.
By using the canonical orbitals, e.i. the eigenfunctions of the Fock operator, the 
Fock matrix is diagonal $F_{ji} = \epsilon_i\delta_{ij}$ and the expression reduces to
\begin{equation}
    \orbital_i = -2\hat{G}_\mu\Big[v_{eff} \orbital_i + 
	\big(\lambda - \epsilon_{i}\big) \orbital_i\Big]
\end{equation}
Furthermore, choosing $\lambda = \epsilon_i$, we get $N$ separated orbital equations similar 
to Eq.~(\ref{eq:int_schrodinger}) (still implicitly coupled through the effective potential)
\begin{equation}
    \orbital_i = -2 \hat{G}_{\mu_i}\Big[v_{eff}\orbital_i\Big]
\end{equation}
with $\mu_i = \sqrt{-2\epsilon_i}$.

\subsubsection{Hartree-Fock}
denoting the total Coulomb potential experienced by the electrons as
\begin{equation}
    v_{coul}(\bs{r}) = v_{nuc}(\bs{r}) + v_{el}(\bs{r})
\end{equation}
\begin{equation}
    \big[-\frac{1}{2}\nabla^2+v_{coul}(\bs{r})-\hat{K}\big]\orbital_i(\bs{r}) 
	    = \sum_j F_{ji} \orbital_j(\bs{r})
\end{equation}
can be rewritten in terms of the integral operator $\hat{G}_\mu$ in the same way as the Kohn-Sham
equations above
\begin{equation}
    \orbital_i = -2\hat{G}_\mu\Big[\big(v_{coul} - \hat{K}\big)\orbital_i + 
	\sum_j \big(\lambda\delta_{ij} - F_{ji}\big) \orbital_j\Big]
\end{equation}
with $\mu = \sqrt{-2\lambda}$. This expression can again be simplified by choosing
canonical orbitals and $\lambda = \epsilon_i$, which leads to separated orbital equations
\begin{equation}
    \orbital_i = -2 \hat{G}_{\mu_i}\Big[\big(v_{coul} - \hat{K}\big)\orbital_i\Big]
\end{equation}
with $\mu_i = \sqrt{-2\epsilon_i}$.

\subsection{Calculation of energy}
We will now assume that the Hartree-Fock or Kohn-Sham equations have been solved to obtain
the orbitals $\orbital_i$ that make up ground state wave function, and their energies 
$\epsilon_i$ (or in general, the Fock matrix $F_{ij}$ in a non-canonical solution), and use
these to calculate the electronic energy of the molecular system. In addition we have the
constant nuclear repulsion energy
\begin{equation}
    \hat{V}_{n-n} = \sum_{I>J} \frac{Z_IZ_J}{\|\bs{R}_I-\bs{R}_J\|}
\end{equation}
The goal of this section is to rewrite the expressions given above into something better 
suited for evaluation in the multiwavelet framework. In particular this means to avoid
the application of the kinetic energy operator, which is a second derivative.

\subsubsection{Hartree-Fock}
The energy of a Slater determinant wave function was given in Eq.~(\ref{eq:det_energy}).
Introducing the orbital occupation $f$ we get a general expression for
both restricted ($f=2$) and unrestricted ($f=1$) Hartree-Fock, where the sums runs over the
number of orbitals
\begin{align}
    E	&= f \sum_i \langle \orbital_i|\hat{h}|\orbital_i \rangle
	    + \frac{f}{2} \sum_i \langle \orbital_i|f\hat{J} - \hat{K}|\orbital_i \rangle\\
	&= f \sum_i \langle \orbital_i|\hat{T}|\orbital_i \rangle
	    + f \sum_i \langle \orbital_i|v_{nuc}|\orbital_i \rangle
	    + \frac{f}{2} \sum_i \langle \orbital_i|v_{el} - \hat{K}|\orbital_i \rangle \\
	&= f \sum_i \langle \orbital_i|\hat{T} - \frac{1}{2}\hat{K}|\orbital_i \rangle
	    + \int \rho(\bs{r})v_{nuc}(\bs{r}) \ud\bs{r}
	    + \frac{1}{2} \int \rho(\bs{r})v_{el}(\bs{r}) \ud\bs{r}
\end{align}
The kinetic energy operator can be avoided by making the following observation:
\begin{align}
    \label{eq:sum_orb_en}
    f \sum_i F_{ii} &= f \sum_i \langle\orbital_i|\hat{T} + v_{nuc} + v_{el} - \hat{K}|\orbital_i\rangle\\
		    &= f \sum_i \langle\orbital_i|\hat{T} - \hat{K}|\orbital_i\rangle
		     + \int \rho(\bs{r})v_{nuc}(\bs{r}) \ud\bs{r}
		     + \int \rho(\bs{r})v_{el}(\bs{r}) \ud\bs{r}
\end{align}
All terms in Eq.~(\ref{eq:sum_orb_en}) are invariant under unitary transformations of the orbitals,
and in a canonical orbital solution the left hand side is the sum over orbital energies $\epsilon_i$.
Comparing the expressions in Eqs.~(\ref{eq:energy_exp_HF}) and (\ref{eq:sum_orb_en}) we see that
the total electronic energy can be calculated as
\begin{equation}
    E = f \sum_i F_{ii} - \frac{1}{2} \int \rho(\bs{r})v_{el}(\bs{r}) \ud\bs{r}
	- \frac{f}{2} \sum_i \langle\orbital_i|\hat{K}|\orbital_i\rangle
\end{equation}
without the need of applying the kinetic energy operator, given the orbitals and Fock matrix that
solves the Hartree-Fock equations.

\subsubsection{Kohn-Sham DFT}
The energy in Kohn-Sham DFT was given through the energy functionals in Eq.~(\ref{eq:KS-energy}) as
\begin{equation}
    E[\rho] = T_s[\rho] + V_{en}[\rho] + J[\rho] + E_{xc}[\rho]
\end{equation}
\begin{equation}
    E = f \sum_i \langle \orbital_i|\hat{T}|\orbital_i \rangle
	+ \int \rho(\bs{r})v_{nuc}(\bs{r}) \ud\bs{r}
	+ \frac{1}{2} \int \rho(\bs{r})v_{el}(\bs{r}) \ud\bs{r}
	+ \int F_{xc} \ud\bs{r}
\end{equation}
\begin{align}
    f \sum_i F_{ii} &= f \sum_i \langle\orbital_i|\hat{T} + v_{eff}|\orbital_i\rangle\\
		    &= f \sum_i \langle\orbital_i|\hat{T}|\orbital_i\rangle
		     + \int \rho(\bs{r})\big(v_{nuc}(\bs{r}) + v_{el}(\bs{r}) + v_{xc}(\bs{r})\big) \ud\bs{r}
\end{align}
\begin{equation}
    E = f \sum_i F_{ii} - \frac{1}{2} \int \rho(\bs{r})v_{el}(\bs{r}) \ud\bs{r}
	+ \int F_{xc} \ud\bs{r} - \int \rho(\bs{r})v_{xc}(\bs{r}) \ud\bs{r}
\end{equation}
where it should be notet that 
\begin{equation}
    E_{xc}[\rho] = \int F_{xc} \ud\bs{r} \neq \int \rho(\bs{r})v_{xc}(\bs{r}) \ud\bs{r}
\end{equation}
