\chapter{Electronic structure theory}
In this chapter we present the equations that govern chemical systems, in particular
the electronic stucture of atoms and molecules. At the moleculear length scale, nature
is most accurately described by the theory of quantum mechanics, where the central
problem is the solution of the non-relativistic Schr\"{o}dinger equation, the 
multi-dimensional partial differential equation that determine the wavefunction 
\Wavefunction that completely describes the quantum system.

\section{The molecular Schr\"{o}dinger equation}\label{sec:schrodinger}
\begin{equation}
    \hat{H}\Wavefunction = E\Wavefunction
\end{equation}
\begin{equation}
    \hat{H} = \hat{T}_{nuc} + \hat{H}_{el}
\end{equation}
\begin{align}
    \Wavefunction &= \wavefunction_{nuc}\wavefunction_{el}\\
    E &= E_{nuc} + E_{el}
\end{align}
\begin{equation}
    \hat{T}_{nuc} = -\sum_I\frac{1}{2m_I} \nabla_I^2
\end{equation}
\begin{equation}
    \hat{H}_{el} = -\sum_i \frac{1}{2}\nabla_i^2 
	+ \sum_{i>j} \frac{1}{|\bs{r}_j-\bs{r}_i|}
	- \sum_{I,i} \frac{Z_I}{|\bs{r}_i-\bs{R_I}|}
        + \sum_{I>J} \frac{Z_IZ_J}{|\bs{R}_I-\bs{R_J}|}
\end{equation}
\begin{equation}
    \hat{H}_{el}\wavefunction_{el} = E_{el}\wavefunction_{el}
\end{equation}
\begin{equation}
    \wavefunction(\bs{x}_1,\bs{x}_2,\dots,\bs{x}_N) = -
    \wavefunction(\bs{x}_2,\bs{x}_1,\dots,\bs{x}_N)
\end{equation}
\begin{equation}
    \label{eq:var_princ}
    \frac{\langle\tilde{\Wavefunction}|\hat{H}|\tilde{\Wavefunction}\rangle}
    {\langle\tilde{\Wavefunction}|\tilde{\Wavefunction}\rangle}
    \geq
    \frac{\langle\Wavefunction_0|\hat{H}|\Wavefunction_0\rangle}
    {\langle\Wavefunction_0|\Wavefunction_0\rangle}
\end{equation}
\begin{equation}
    \hat{h}_0 = \sum_{I<J} \frac{Z_IZ_J}{|\bs{R}_I-\bs{R}_J|}
\end{equation}

\section{Hartree-Fock Theory}\label{sec:HFT}
\subsection{Slater determinant}
\begin{equation}
    \wavefunction(\bs{x}_1,\bs{x}_2,\dots,\bs{x}_N) = \sum_m c_m 
	\orbital_1^m(\bs{x}_1)
	\orbital_2^m(\bs{x}_2)\cdots
	\orbital_N^m(\bs{x}_N)
\end{equation}
\begin{equation}
    \begin{split}
    \wavefunction(\bs{x}_1,\bs{x}_2,\dots,\bs{x}_N) = 
    |\orbital_1\orbital_2\cdots\orbital_N\rangle \mydef \frac{1}{\sqrt{N}} 
    \left|
    \begin{array}{cccc}
	\orbital_1(\bs{x}_1)	& \orbital_1(\bs{x}_2)	& \cdots & \orbital_1(\bs{x}_N)\\
	\orbital_2(\bs{x}_1)	& \orbital_2(\bs{x}_2)	& \cdots & \orbital_2(\bs{x}_N)\\
	\vdots			& \vdots		& \ddots & \vdots\\
	\orbital_N(\bs{x}_1)	& \orbital_N(\bs{x}_2)	& \cdots & \orbital_N(\bs{x}_N)
    \end{array}
    \right|
    \end{split}
\end{equation}
\begin{equation}
    \hat{h}_i = \bigg(-\frac{1}{2}\nabla_i^2
	-\sum_I \frac{Z_I}{|\bs{r}_i-\bs{R}_I|}\bigg)
\end{equation}
\begin{equation}
    \hat{g}_{ij} = \frac{1}{|\bs{r}_j-\bs{r}_i|}
\end{equation}
\begin{equation}
    \hat{H} = \sum_i \hat{h}_i + \sum_{i<j} \hat{g}_{ij}
\end{equation}
The energy of a Slater determinant wave function is evaluated as the expectation value
of the Hamiltonian, and is given by
\begin{align}
    E[\wavefunction] &=
    \langle\orbital_1\orbital_2\cdots\orbital_N|\hat{H}|
    \orbital_1\orbital_2\cdots\orbital_N\rangle\\
    &= \sum_{i=1}^N \langle \orbital_i |\hat{h}| \orbital_i \rangle +
    \frac{1}{2} \sum_{i=1}^N\sum_{j=1}^N \Big(
    \langle \orbital_i\orbital_j |\hat{g}| \orbital_i\orbital_j\rangle -
    \langle \orbital_i\orbital_j |\hat{g}| \orbital_j\orbital_i\rangle\Big)
\end{align}
where the one- and two-electron integrals are
\begin{align}
    \langle \orbital_i|\hat{O}|\orbital_j \rangle &= \int\orbital_i^{\ast}(\bs{x}_1)
    \hat{O}\orbital_j(\bs{x}_1) \ud \bs{x}_1\\
    \langle \orbital_i\orbital_j|\hat{O}|\orbital_k\orbital_l \rangle &= \int \int 
    \orbital_i^{\ast}(\bs{x}_1)\orbital_j^{\ast}(\bs{x}_2)
    \hat{O}\orbital_k(\bs{x}_1)\orbital_l(\bs{x}_2) \ud \bs{x}_1 \ud \bs{x}_2
\end{align}
Introducing the Coulomb $\hat{J}=\sum_i\hat{J}_i$ and exchange $\hat{K}=\sum_i\hat{K}_i$
operators, which are defined through their action on an arbitrary function $f$ 
\begin{align}
    \hat{J}_j f(\bs{x}_1) &= \bigg(\int \frac{\orbital_j^{\ast}(\bs{x}_2)\orbital_j(\bs{x}_2)}
	{\|\bs{r}_2-\bs{r}_1\|}\ud\bs{x}_2\bigg) f(\bs{x}_1)\\
    \hat{K}_j f(\bs{x}_1) &= \bigg(\int \frac{\orbital_j^{\ast}(\bs{x}_2)f(\bs{x}_2)}
	{\|\bs{r}_2-\bs{r}_1\|}\ud\bs{x}_2\bigg) \orbital_j(\bs{x}_1)
\end{align}
we arrive at the expression for the total electronic energy of a Slater determinant
\begin{equation}
    E[\wavefunction] = 
    \sum_{i=1}^N \langle \orbital_i |\hat{h}| \orbital_i \rangle +
    \frac{1}{2} \sum_{i=1}^N 
    \langle \orbital_i |\hat{J}-\hat{K}| \orbital_i\rangle
\end{equation}
The Hartree-Fock energy $E_{HF}$ is obtained by minimizing the energy with respect 
to orbital variations
\begin{equation}
    E_{HF} = \mymin{\wavefunction}\ E[\wavefunction]
\end{equation}
following the variational principle of Eq.~(\ref{eq:var_princ}), under the constraint 
that the orbitals remain orthonormal $\langle\orbital_i|\orbital_j\rangle = \delta_{ij}$.

\subsection{Hartree-Fock equations}
\begin{equation}
    \orbital_i^\sigma(\bs{r},s) = \orbital_i(\bs{r})\sigma(s), \qquad \sigma = \alpha, \beta
\end{equation}
\begin{equation}
    \wavefunction(\bs{x}_1,\bs{x}_2,\dots,\bs{x}_N) = 
    |\orbital_1^\alpha\orbital_1^\beta\cdots\orbital_{N/2}^\alpha\orbital_{N/2}^\beta\rangle
\end{equation}
The Hamiltonian for a closed-shell system is 
\begin{equation}
    \hat{F}_{HF} = \hat{h} + \sum_j^{N/2} \Big(2\hat{J}_j - \hat{K}_j\Big)
\end{equation}
and is usually referred to as the Fock operator.
\begin{equation}
    v_{nuc}(\bs{r}) = \sum_{I} \frac{Z_I}{\|\bs{r} - \bs{R}_I\|}
\end{equation}
\begin{equation}
    v_{el}(\bs{r}) = \sum_j^{N/2} 2\hat{J}_j = 2 \sum_j^{N/2} \int \frac{|\orbital_j(\bs{r}')|^2}
	{\|\bs{r} - \bs{r}'\|} \ud \bs{r}'
\end{equation}
\begin{equation}
    \hat{K}\orbital_i(\bs{r}) = \sum_j^{N/2} \hat{K}_j \orbital_i(\bs{r}) 
	= \sum_j^{N/2} \orbital_j (\bs{r}) \int \frac{\orbital_j^{\ast}(\bs{r}')\orbital_i(\bs{r}')}
	    {\|\bs{r} - \bs{r}'\|} \ud \bs{r}'
\end{equation}
leads to the canonical Hartree-Fock equations
\begin{equation}
    \Big[-\frac{1}{2}\nabla^2 + v_{nuc}(\bs{r}) + v_{el}(\bs{r}) - \hat{K}\Big]\orbital_i(\bs{r}) 
	= \epsilon_i \orbital_i(\bs{r})
\end{equation}
where the Hartree-Fock wave function is the Slater determinant constructed from the $N/2$ lowest
energy eigenfunctions of the Fock operator, each appearing twice ($\alpha$ and $\beta$ spins).

\subsection{Spin-unrestricted Hartree-Fock}
\begin{equation}
    v_{el}(\bs{r}) = v_{el}^\alpha (\bs{r}) + v_{el}^\beta (\bs{r})
\end{equation}
The exchange operator runs over all orbitals of a given spin
\begin{equation}
    \hat{K}^\sigma\orbital_i^\sigma(\bs{r}) 
	= \sum_j^{N_\sigma} \orbital_j^\sigma (\bs{r}) 
	    \int \frac{\orbital_j^{\sigma\ast}(\bs{r}')\orbital_i^\sigma(\bs{r}')}
	    {\|\bs{r} - \bs{r}'\|} \ud \bs{r}', \qquad \sigma = \alpha, \beta
\end{equation}
\begin{equation}
    \hat{F}_{HF}^\sigma = -\frac{1}{2}\nabla^2 + v_{nuc}(\bs{r}) + v_{el}(\bs{r}) - \hat{K}^\sigma 
\end{equation}
and the unrestricted wave function is the determinant constructed by the $N_\alpha$ and 
$N_\beta$ lowest energy eigenfunctions of the operators $\hat{F}_{HF}^\alpha$ and 
$\hat{F}_{HF}^\beta$, respectively
\begin{equation}
    \wavefunction(\bs{x}_1,\bs{x}_2,\dots,\bs{x}_N) = 
    |\orbital_1^\alpha\orbital_1^\beta\cdots\orbital_{N_\alpha}^\alpha\orbital_{N_\beta}^\beta\rangle
\end{equation}

\section{Density Functional Theory} \label{sec:DFT}
\begin{equation}
    \rho(\bs{r}_1) = N \int |\wavefunction(\bs{x}_1,\bs{x}_2,\dots,\bs{x}_N)|^2 
	\ud s_1 \ud \bs{x}_2 \cdots \ud \bs{x}_N
\end{equation}
\begin{equation}
    E[\rho] = T[\rho] + V_{ne}[\rho] + V_{ee}[\rho]
\end{equation}
\begin{align}
    V_{ne}[\rho] &= \int \rho(\bs{r})v_{nuc}(\bs{r}) \ud \bs{r}\\
    F[\rho] &= T[\rho] + V_{ee}[\rho]
\end{align}
\begin{equation}
    E_0 = \mymin{\rho}\ E[\rho]
\end{equation}
with the constraints that the density is everywhere positive and integrates to the number
of electrons.

\subsection{Kohn-Sham equations}
\begin{equation}
    \rho(\bs{r}) = 2 \sum_i^{N/2} | \orbital_i(\bs{r})|^2
\end{equation}
\begin{equation}
    F[\rho] = T_s[\rho] + J[\rho] + E_{xc}[\rho]
\end{equation}
\begin{equation}
    T_s[\rho] = \sum_i^{N/2} \langle \orbital_i | -\frac{1}{2}\nabla^2 | \orbital_i \rangle
\end{equation}
\begin{equation}
    J[\rho] = \frac{1}{2} \int \rho(\bs{r})v_{el}(\bs{r}) \ud \bs{r}
\end{equation}
\begin{equation}
    E_{xc}[\rho] = T[\rho] - T_s[\rho] + V_{ee}[\rho] - J[\rho]
\end{equation}
\begin{equation}
    E[\rho] = V_{en}[\rho] + J[\rho] + E_{xc}[\rho] + T_s[\rho]
\end{equation}
leads to the Euler equation
\begin{equation}
    \mu = v_{eff}(\bs{r}) + \frac{\delta T_s[\rho]}{\delta \rho(\bs{r})}
\end{equation}
where the chemical potential $\mu$ is a Lagrange multiplier that fixes the number of 
electrons and the effective potential is given by
\begin{align}
    v_{eff}(\bs{r}) 
	&= \frac{\delta V_{en}[\rho]}{\delta \rho(\bs{r})}
	+ \frac{\delta J[\rho]}{\delta \rho(\bs{r})}
	+ \frac{\delta E_{xc}[\rho]}{\delta \rho(\bs{r})}\\
	&= v_{nuc}(\bs{r}) + v_{el}(\bs{r}) + v_{xc}(\bs{r})
\end{align}
The Fock operator for the system of non-interacting electrons influenced by an effective
potential is given simply as
\begin{equation}
    \hat{F}_{KS} = -\sum_i^{N/2} \frac{1}{2}\nabla_i^2 + \sum_i^{N/2} v_{eff}(\bs{r}_i)
\end{equation}
and is also called the Kohn-Sham operator. As there are no couplings between the electrons, 
this operator is separable, and the exact wave function is given by a single determinant 
constructed by the $N/2$ lowest energy eigenfunctions of the Kohn-Sham operator
\begin{equation}
    \Big[-\frac{1}{2}\nabla^2 + v_{nuc}(\bs{r}) + v_{el}(\bs{r}) + v_{xc}(\bs{r})\Big] 
	\orbital_i(\bs{r}) = \epsilon_i \orbital_i(\bs{r})
\end{equation}
where each orbital appears in the determinant with $\alpha$ and $\beta$ spins.

\subsection{The Exchange-Correlation potential}
The exchange-correlation energy functional is given as an integral over an energy
density $F_{xc}$
\begin{equation}
    E_{xc}[\rho] = \int F_{xc}\ud \bs{r}
\end{equation}
In the local density approximation (LDA) the energy density is a function of the density 
alone $F_{xc}(\rho)$, in the generalized gradient approximation (GGA) it is a function of
the density and its gradient $F_{xc}(\rho, |\nabla\rho|)$, while in meta-GGA's, higher order 
derivatives are introduced $F_{xc}(\rho, |\nabla\rho|, \nabla^2\rho, \cdots)$. Hybrid 
functionals, such as the very popular B3LYP\cite{B3LYP}, are GGA's with a certain amount of
exact Hartree-Fock exchange, evaluated as in Eq.~(\ref{eq:HFX}) using Kohn-Sham orbitals.

The exchange-correlation potential was implicitly defined in Eq.~(\ref{eq:XXX}) as the
functional derivative of the exchange-correlation energy with respect to the density
\begin{equation}
    v_{xc} = \frac{\delta E_{xc}[\rho]}{\delta \rho} 
	= \frac{\delta}{\delta \rho} \int F_{xc} \ud \bs{r}
\end{equation}
which in LDA reduces to
\begin{equation}
    v_{xc}^{LDA} = \frac{\partial F_{xc}}{\partial \rho}
\end{equation}
while a second derivative of $F_{xc}$ is required for GGA's
\begin{equation}
    v_{xc}^{GGA} = \frac{\partial F_{xc}}{\partial \rho} - 
	\nabla\cdot\frac{\partial F_{xc}}{\partial\nabla\rho}
\end{equation}

\subsection{Spin-unrestricted Kohn-Sham}
The extension to spin-unrestricted and open-shell systems is straightforward. In this case the
$\alpha$ and $\beta$ electrons occupy different spatial orbitals, $\orbital^\alpha$ and 
$\orbital^\beta$, and we define the corresponding spin densities
\begin{equation}
    \rho^{\sigma}(\bs{r}) = 
	\sum_{i=1}^{N_{\sigma}} |\phi_i^{\sigma}(\bs{r})|^2, \qquad \sigma = \alpha, \beta
\end{equation}
All spin effects are included in the exchange-correlation potential, which in this case will 
depend on the spin densities and possibly their gradients, and the $\alpha$ and $\beta$ electrons 
will experience different effective potentials
\begin{equation}
    v_{eff}^{\sigma} (\bs{r}) = 
	v_{nuc}(\bs{r}) + v_{el}(\bs{r}) + v_{xc}^{\sigma}(\bs{r}), \qquad \sigma = \alpha, \beta
\end{equation}
The nuclear potential is the same as before, whereas the electronic potential is obtained
from the total electron density
\begin{equation}
    v_{el}(\bs{r}) = \int \frac{\rho^\alpha(\bs{r}') + \rho^\beta(\bs{r}')}
	{\|\bs{r}-\bs{r}'\|} \ud \bs{r}'
\end{equation}
This leads to different Kohn-Sham operators for the different spins
\begin{equation}
    \hat{F}_{KS}^\sigma = -\sum_i^{N_\sigma} \frac{1}{2}\nabla_i^2 + 
	\sum_i^{N_\sigma} v_{eff}^\sigma(\bs{r}_i), \qquad \sigma = \alpha, \beta
\end{equation}
and a single determinant wave function is constructed by the $N_\alpha$ and $N_\beta$ 
lowest energy eigenfunctions of the operators $\hat{F}_{KS}^\alpha$ and 
$\hat{F}_{KS}^\beta$, respectively.

\section{Basis sets in computational chemistry}

\section{Integral formulation}
Although the differential form in Eq.~(\ref{eq:diff_schrodinger}) is most 
commonly used, it is not very well suited for solution in the multiwavelet 
basis \cite{Harrison}, and differential operators (in particular higher 
order operators) should be avoided in order to maintain high accuracy. 
Kalos \cite{Kalos} showed in 1962 that the Schr\"{o}dinger equation can 
be formulated as an integral equation, using the Helmholtz Green's functions 
of Eq.~(\ref{eq:greens_kernel})

\subsection{One-electron systems}
\begin{align}
    \big[-\frac{1}{2}\nabla^2+v_{nuc}(\bs{r})\big]\wavefunction(\bs{r}) &= E \wavefunction(\bs{r})\\
    \big[-\nabla^2 - 2E\big]\wavefunction(\bs{r}) &= -2v_{nuc}(\bs{r}) \wavefunction(\bs{r})\\
    \wavefunction(\bs{r}) &= -2\int H_{\mu}(\bs{r}-\bs{r}')v_{nuc}(\bs{r}') 
	\wavefunction(\bs{r}') \ud \bs{r}'\\
    \label{eq:int_schrodinger}
    \wavefunction &= -2 \hat{G}_{\mu}\big[v_{nuc}\wavefunction\big]
\end{align}
with $\mu = \sqrt{-2E}$.


\subsection{Many-electron systems}

\begin{equation}
    F_{ij} = \langle \orbital_i | \hat{F} | \orbital_j \rangle
\end{equation}
\begin{equation}
    \hat{F}| i \rangle	
	= \Big[\sum_j |j \rangle\langle j|\Big] \hat{F}|i \rangle 
	= \sum_j F_{ji} |j \rangle
\end{equation}

\subsubsection{Kohn-Sham DFT}
\begin{align}
    \big[-\frac{1}{2}\nabla^2+v_{eff}(\bs{r})\big]\orbital_i(\bs{r}) 
	    &= \sum_j F_{ji} \orbital_j(\bs{r})\\
    \big[-\nabla^2 - 2\lambda\big]\orbital_i(\bs{r}) &= -2\Big[v_{eff}(\bs{r}) \orbital_i(\bs{r})
	    + \sum_j \big(\lambda\delta_{ij} - F_{ji}\big) \orbital_j(\bs{r})\Big]\\
    \orbital_i &= -2\hat{G}_\mu\Big[v_{eff} \orbital_i + 
	\sum_j \big(\lambda\delta_{ij} - F_{ji}\big) \orbital_j\Big]
\end{align}
where $\mu = \sqrt{-2\lambda}$. This general expression can be simplified in many ways.
By using the canonical orbitals, e.i. the eigenfunctions of the Fock operator, the 
Fock matrix is diagonal $F_{ji} = \epsilon_i\delta_{ij}$ and the expression reduces to
\begin{equation}
    \orbital_i = -2\hat{G}_\mu\Big[v_{eff} \orbital_i + 
	\big(\lambda - \epsilon_{i}\big) \orbital_i\Big]
\end{equation}
Furthermore, choosing $\lambda = \epsilon_i$, we get $N$ separated orbital equations similar 
to Eq.~(\ref{eq:int_schrodinger}) (still implicitly coupled through the effective potential)
\begin{equation}
    \orbital_i = -2 \hat{G}_{\mu_i}\big[v_{eff}\orbital_i\big]
\end{equation}
with $\mu_i = \sqrt{-2\epsilon_i}$.


As for the one-electron problem, we want to rewrite the differential 
equations in Eq.~(\ref{eq:KS-diff}) into integral form in order to be 
solved using the multiwavelet basis. This can be done straightforwardly
by the procedure of section \ref{sec:1_orb_int} for each orbital separately, 
\begin{equation}
    \label{eq:KS-int}
    \varphi_i(r) = -2\int H_{\mu_i}(r-r')V_{eff}(r') \varphi_i(r') \ud r'
\end{equation}
and the equations are again solved iteratively
\begin{equation}
    \label{eq:KS-iter}
    \tilde{\varphi}^{n+1}_i = -2 \hat{H}^n\big[V^n_{eff}\varphi^n_i\big]
\end{equation}
although care must be taken in order to keep orthonormality between the 
orbitals. A simple iteration of all $N$ orbitals separately will not 
conserve the canonical character of the input orbitals, and all orbitals 
will eventually collapse into the lowest energy eigenfunction, unless 
orthogonality is explicitly enforced. There are different ways in which 
one can keep orthonormality, and these are discussed below. Note also the
iteration label on the effective potential, as this depends on the orbitals 
and must be updated in each iteration.

\subsubsection{Hartree-Fock}
\subsection{Calculation of Fock matrix}
\subsection{Calculation of energy}

